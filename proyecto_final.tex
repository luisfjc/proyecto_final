\documentclass{article}
\usepackage[utf8]{inputenc}
\usepackage[spanish]{babel}

\begin{document}
	\title{\textbf{\Huge{EVOLUCIÓN DE LOS MATRIMONIOS HOMOSEXUALES Y SUS RUPTURAS}}}
	\author{José Manuel Jiménez Cabello \\ Diego Becerril Ruíz}
	\maketitle
	\newpage
	
	\tableofcontents
	\newpage
	
	\section{Resumen} 
	

\textbf{Link del repositorio:} \\

	\footnote[1]{A fin de dedicarme exclusivamente a la elaboración del código Latex y la utilización de Git, utilizaré el presente artículo, para lo cual poseo autorización expresa de los autores}
El presente trabajo muestra la evolución de los matrimonios homosexuales desde su legalización en el año 2005, mediante la ley 13/2005 y como son, en su caso, las posteriores disoluciones. Una vez superada la legalización las parejas homosexuales se han quintuplicado, siendo la presencia del matrimonio entre homosexuales la opción mayoritaria, más entre mujeres que entre hombres. Además el número de matrimonios entre hombres y mujeres ha evolucionado hasta ser prácticamente igual. En cuanto a las disoluciones, los matrimonios homosexuales cuentan con un alto grado de consenso, siendo esta una característica que está presente desde el origen de su existencia y que además muestra una gran estabilidad. Algunos factores como la nacionalidad o la tenencia o no de hijos menores, son claves para conocer como se produce la ruptura.
\paragraph{Palabras clave:}
 \textit{Ruptura, divorcio, matrimonio, conflicto, consenso, homosexual.}
\section{Introducción}
El presente trabajo muestra la evolución de los matrimonios homosexuales desde su legalización en el año 2005, mediante la ley 13/2005 y como son, en su caso, las posteriores disoluciones. Una vez superada la legalización las parejas homosexuales se han quintuplicado, siendo la presencia del matrimonio entre homosexuales la opción mayoritaria, más entre mujeres que entre hombres. Además el número de matrimonios entre hombres y mujeres ha evolucionado hasta ser prácticamente igual. En cuanto a las disoluciones, los matrimonios homosexuales cuentan con un alto grado de consenso, siendo esta una característica que está presente desde el origen de su existencia y que además muestra una gran estabilidad. Algunos factores como la nacionalidad o la tenencia o no de hijos menores, son claves para conocer como se produce la ruptura.
\section{Estado del arte}
\section{Imágenes y tablas}
\section{Fórmulas}

\end{document}